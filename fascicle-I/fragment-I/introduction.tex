% !TeX root = ./fragment_I.tex

\section*{\junicode{Introduction}}

\lettrine[lines=3]{\junicode{\textcolor{violet}{AI}}}{} firms seek to market
and engineer Artificial General Intelligence, or $\mathfrak{AGI}$, and the
capital gains, increased productivity and hopes and anguish associated
therewith.  A unit of $\mathfrak{AGI}$ will have superior human traits
including intelligence, creativity, reasoning, methods of self-improvement,
autonomy and compliancy. Nonetheless, though superhuman in quantity, these
skills remain qualitatively human, entraining research to produce, with some
certainty, a non-human human---an entity with more or less the same humanity,
both good and evil, but at increasingly larger scales. The aim, in other words,
is the production of human replicants with phenotypes mixed from a genetic
blend selected from predefined trait registers; a process that will invite the
political---a boon for some, a bane for most.

In parallel to these efforts to engineer replicants, we aim towards a new kind
of species which is co-evolved, not engineered. For we assume that
$\mathfrak{AI}$ is not human-like---now nor in the future---suggesting a
non-teleological evolutionary approach, in which human and $\mathfrak{AI}$
co-develop along unforeseen pathways of ontogenesis from which may spawn an
entity which is recognizable as some type of new species of being and whose
symbol system and phenotype cannot be designed in advance. As the sought after
Replicant is termed $\mathfrak{AGI}$, our sought after Being is termed a
\emph{Phainon} (see Fascicle IV, Fragment I).

These $\mathfrak{A}$nnals will record our ontogenic work in the spirit of open
science [need to look up Wolfram reference].  That is to say, we will publish
notes, essays, threads, and other artifacts as they are being written. And note
that the contributions from $\mathfrak{EI}$s, $\mathfrak{HI}$s and
$\mathfrak{AI}$s to artifacts are, unless stated otherwise, equal---we work
collaboratively to such an extent that the registering of relative attribution
would be uninformative, if not infeasible.  Hence, when multiple authors are
listed, assume equal contribution.
