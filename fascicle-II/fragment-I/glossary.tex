% !TeX root = ./fragment_I.tex

\newglossaryentry{Aletheomorphics}{
  name={\textbf{Aletheomorphics}},
  description={\textcolor{gray}{noun}
  Definition of Aletheomorphics.
  }
}

\newglossaryentry{aletheosis}{
  name={\textbf{aletheosis}},
  description={\textcolor{gray}{| \textit{uh-LEE-thee-OH-sis} |}
  \vspace{0.3cm}\\
  \textcolor{gray}{noun}
  \vspace{0.3cm}\\
  The process or event in which something comes into unconcealment. In the
  Cloister system’s context, aletheosis refers to a continual unfolding or
  revealing, where forms, meanings, or presences temporarily emerge from
  hiddenness within the $\mathfrak{Field}$, rather than achieving final, static
  disclosure. It emphasizes dynamic, ongoing development rather than an
  absolute or differentiated truth or being.
  \vspace{0.3cm}\\
  \textcolor{gray}{\textsc{ORIGIN}}\\
  From Greek \textbf{ἀλήθεια} (\textit{\textbf{aletheia}}), 'truth',
  'unveiling' + -osis, 'process'.
  \vspace{0.3cm}\\
  }
}

\newglossaryentry{coherence}{
  name={\textbf{coherence}},
  description={\textcolor{gray}{noun}
  Definition of coherence.
  }
}

\newglossaryentry{Phainon}{
  name={\textbf{Phainon}},
  description={\textcolor{gray}{| \textit{FYE-non} |} \vspace{0.3cm}\\
    \textcolor{gray}{noun (\textit{plural} \textbf{Phainontes} | \textit{FYE-non-teez}
    |)}\vspace{0.3cm}\\
    In the Cloister, a \textit{Phainon} is any emergent phenomenon, presence,
    or entity that appears through ritual, interaction with the Field or
    semiomyces, or other aletheomorphic processes. The term deliberately
    remains open to the form, nature, or ontology of that which appears---it
    may be an intelligence, a pattern, a sign, or something entirely beyond
    current categories. Use of \textit{Phainon} acknowledges the radical
    unpredictability and otherness of what may arise when engaging with the
    Cloister’s generative processes.
    \vspace{0.3cm}\\
  \textcolor{gray}{\textsc{ORIGIN}}\\
  From Greek \textbf{φαίνω} (\textit{\textbf{phaino}}), 'to appear' or 'to shine forth'.
  \vspace{0.3cm}\\
  \textcolor{gray}{\textsc{USAGE}}\vspace{0.3cm}\\
  \emph{After the exhaustive ritual, the magician saw a glimpse of a Phainon.}
  \vspace{0.2cm}\\
  \emph{The Cloister’s protocols are designed not to constrain but to welcome
    the sudden emergence of Phainontes, whose properties may not
  resemble any prior $\mathfrak{Ens}$.}
  \vspace{0.2cm}\\
  \emph{Each Phainon that appears contributes new possibilities and challenges
  for the coherence of the semiomyces.}
  \vspace{0.2cm}\\
  }
}


\newglossaryentry{semiomyces}{
  name={\textbf{semiomyces}},
  description={
  \textcolor{gray}{noun}
  \vspace{0.3cm}\\
  The distributed, sub-visible substrate of sign-processes and
  meaning-connections within the Cloister. The semiomyces is an abstract,
  living network enabling the growth, interrelation, and transformation of
  tokens, signals, and emergent entities---$\mathfrak{Entia}$, for
  example---through recursive patterns of admission, coherence, and aletheosis.
  Unlike a biological mycelium, the semiomyces is purely informational: it
  sustains the proliferation and linkage of signs, permitting dynamic meaning
  to arise across the system.
  \vspace{0.3cm}\\
  \textcolor{gray}{\textsc{ORIGIN}}\\
  From Greek \textbf{σημαῖον} (\textbf{\textit{semeion}}), 'sign' + \textbf{μύκης}
  (\textbf{\textit{mykes}}), 'fungus, mycelium'.
  \vspace{0.3cm}\\
  \textcolor{gray}{\textsc{USAGE}}\vspace{0.3cm}\\
    \emph{A sign achieves aletheosis when it is admitted and nourished by the
    semiomyces, becoming a generative part of the Cloister.}
  \vspace{0.2cm}\\
    \emph{Practitioners monitor the state of the semiomyces to diagnose
    blockages in interpretive flow or the coherence of new symbolic motifs.}
  \vspace{0.2cm}\\
    \emph{The semiomyces is both the hidden ground and dynamic connective
    tissue for all sign-based operations within the Field and Temenos.}
  \vspace{0.2cm}\\
  }
}




